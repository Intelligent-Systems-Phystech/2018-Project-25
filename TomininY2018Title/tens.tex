\documentclass[parskip]{scrartcl}
\usepackage[margin=15mm]{geometry}
\usepackage{tikz}

\newcommand{\tikzcuboid}[4]{% width, height, depth, scale
\begin{tikzpicture}[scale=#4]
\foreach \x in {0,...,#1}
{   \draw (\x ,0  ,#3 ) -- (\x ,#2 ,#3 );
    \draw (\x ,#2 ,#3 ) -- (\x ,#2 ,0  );
}
\foreach \x in {0,...,#2}
{   \draw (#1 ,\x ,#3 ) -- (#1 ,\x ,0  );
    \draw (0  ,\x ,#3 ) -- (#1 ,\x ,#3 );
}
\foreach \x in {0,...,#3}
{   \draw (#1 ,0  ,\x ) -- (#1 ,#2 ,\x );
    \draw (0  ,#2 ,\x ) -- (#1 ,#2 ,\x );
}

\end{tikzpicture}
}

\newcommand{\tikzcube}[2]{% length, scale
\tikzcuboid{#1}{#1}{#1}{#2}
}

\begin{document}
Полученный интегральный дескриптор $\textbf{g}$ является объединением всех блочных дейскрипторов. Признаковое пространство G имеет размерность ${2(I-1)}{2(J-1}){l}$. Для востановления пространственной смежности ячеек нужно перейти в новое трехмерное тензорное пространство с размерностью  ${2(I-1)}\times{2(J-1})\times{l}$. \\
Изначально вектор g имеет вид:\\

$\underbrace{\tikzcuboid{12}{1}{1}{0.5}}_{4\ell (1)}$ 
$\underbrace{\tikzcuboid{12}{1}{1}{0.5}}_{4\ell (2)}$ $\cdots$ 
$\underbrace{\tikzcuboid{12}{1}{1}{0.5}}_{4\ell (\textbf{I}\times \textbf{J})}$

Далее проводится операция с каждым вектором размерности $4 \ell$\\

$\underbrace{\tikzcuboid{12}{1}{1}{0.5}}_{4\ell (i)}$ $\Rightarrow$ $\tikzcuboid{2}{2}{4}{0.5}$

После итеррации

$\underbrace{\tikzcuboid{2}{2}{4}{0.5} \cdot \cdot \cdot \tikzcuboid{2}{2}{4}{0.5}}_{\textbf{I}(1)}$ 
$\underbrace{\tikzcuboid{2}{2}{4}{0.5} \cdot \cdot \cdot \tikzcuboid{2}{2}{4}{0.5}}_{\textbf{I}(2)}$  $\cdots$ 
$\underbrace{\tikzcuboid{2}{2}{4}{0.5} \cdot \cdot \cdot \tikzcuboid{2}{2}{4}{0.5}}_{\textbf{I}(\textbf{J})}$ 

Меняется расположение блоков

$\underbrace{\tikzcuboid{2}{2}{4}{0.5} \cdot \cdot \cdot \tikzcuboid{2}{2}{4}{0.5} \cdot \cdot \cdot \tikzcuboid{2}{2}{4}{0.5} \cdot \cdot \cdot \tikzcuboid{2}{2}{4}{0.5}}_{\textbf{I}(1)}$ \\
$\underbrace{\tikzcuboid{2}{2}{4}{0.5} \cdot \cdot \cdot \tikzcuboid{2}{2}{4}{0.5} \cdot \cdot \cdot \tikzcuboid{2}{2}{4}{0.5} \cdot \cdot \cdot \tikzcuboid{2}{2}{4}{0.5}}_{\textbf{I}(2)}$ \\
  $\cdots \cdots \cdots \cdots \cdots \cdots \cdots \cdots \cdots \cdots \cdots \cdots \cdots \cdots \cdots \cdots \cdots \cdots  $ \\
$\underbrace{\tikzcuboid{2}{2}{4}{0.5} \cdot \cdot \cdot \tikzcuboid{2}{2}{4}{0.5} \cdot \cdot \cdot \tikzcuboid{2}{2}{4}{0.5} \cdot \cdot \cdot \tikzcuboid{2}{2}{4}{0.5}}_{\textbf{I}(\textbf{J})}$ \\

При объединении всех блоков получается

$\underbrace{\tikzcuboid{20}{14}{4}{0.5}}_{\textbf{I}}$ 



\end{document}