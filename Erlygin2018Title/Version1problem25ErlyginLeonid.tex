

\documentclass[a4paper,12pt]{article}
\usepackage[T1,T2A]{fontenc}
\usepackage[utf8]{inputenc}
\usepackage[english,russian]{babel}
\usepackage{tikz}
\usepackage{mathtext}
\usepackage{graphicx}
\graphicspath{{image/}}
\DeclareGraphicsExtensions{.pdf,.png,.jpg,.odg}
\usepackage{listings}
\usepackage{amsmath,amsfonts,amssymb,amsthm,mathtools} 


\author{Леонид Ерлыгин 773}
\title{Анотация к проекту 25} 
\date{\today}
\begin{document}
\maketitle
\newpage
Данная работа описывает новый алгоритм автоматического детектирования и распознавания объектов на изображениях и видео. Алгоритм представляет собой дескриптор, который обладает меньшей спецефичностью, чем аналогичные алгоритмы(AdaBoost, c Haar-дескрипторами) и он позволяет различать разные ракурсы объекта. В качестве целевых объектов можно использовать: лица людей, автомобили. Данная задача остро востребована для нахождения автомобилей, пешеходов и сцен с плохими условаиями освещения.
\end{document}
 

