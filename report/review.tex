\documentclass[a4paper,12pt]{article}
\usepackage[T2A]{fontenc}			
\usepackage[utf8]{inputenc}	
\usepackage{tikz}		
\usepackage[english,russian]{babel}	
\usepackage{amsmath,amsfonts,amssymb,amsthm,mathtools} 
\usepackage{wasysym}
\title{Автоматическое детектирование и распознавание объектов на изображениях}
\usepackage{graphicx}
\graphicspath{}
\DeclareGraphicsExtensions{.pdf,.png,.jpg}
\begin{document}
\maketitle
\newpage

	Работа посвящена обнаружению объектов на изображениях, другими словами, разделению изображений на два класса: содержащих искомый объект и несодержащих. Признаковое описание изображения самого объекта строится с помощью дескрипторов. Наиболее известные из них: HOG, Haar, LBP, SIFT. Одной из целей работы является усовершенствование существующих дескрипторов или создание нового для получения лучшего результата на ROC и DET кривых при заданном методе классификации. Кроме того, проблемой существующих дескрипторов является чувствительность к ракурсу съемки. Для её решения предполагается разработка нового метода построения тексутрных признаков на основе данных дескрипторов, использование решающих деревьев.

	Другой целью является построение устойчивого к ракурсу и освещению классификатора для детектирования объектов и их деталей с использованием решающих деревьев, а также создание классификатора для мимики лица на основе 3D модели, построенной с помощью свёрточной нейронной сети.
	
	Решение задачи востребовано в распозновании частично загороженных объектов, а также объектов, находящихся в условиях плохого освещения. Наиболее извстными примерами могут служить: детектирование пешеходов, лиц, автомобилей и их особенностей.         


\end{document}